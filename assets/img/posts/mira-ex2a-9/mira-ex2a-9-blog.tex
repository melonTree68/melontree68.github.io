% xelatex
\documentclass[11pt,a4paper]{article}
\usepackage{geometry}
\usepackage{amsmath,amssymb,amsfonts,amsthm}
\usepackage[dvipsnames,svgnames]{xcolor}
\usepackage{enumerate,enumitem}
\usepackage{thmtools}
\usepackage{fancyhdr}
\usepackage{ragged2e}
\usepackage{titling}
\usepackage{hyperref}
\usepackage{graphicx}
\usepackage{float}
\usepackage{subcaption}


% ---------- Preamble ----------
\pagestyle{fancyplain} % Make all pages in the document conform to the custom headers and footers
\fancyhead{} % No page header
\fancyfoot[L]{}
\fancyfoot[C]{}
\fancyfoot[R]{\thepage}
\renewcommand{\headrulewidth}{0pt} % Remove header underlines
\renewcommand{\footrulewidth}{0pt} % Remove footer underlines
\geometry{
    textwidth=160mm,
    top=25mm,
    bottom=25mm,
}
\setlist[enumerate]{
    label=\textnormal{(\arabic*)},
    topsep=.5em,
    itemsep=0em,
}
\setlist[itemize]{
    topsep=.5em,
    itemsep=0em,
}


% ---------- Theorem environments ----------
\theoremstyle{plain} % default
\newtheorem{thm}{Theorem}
\newtheorem*{cor}{Corollary}
\newtheorem*{prop}{Proposition}
\newtheorem*{lem}{Lemma}
\newtheorem*{conj}{Conjecture}
\newtheorem*{quest}{Question}

\theoremstyle{definition}
\newtheorem*{defn}{Definition}
\newtheorem*{prob}{Problem}

\theoremstyle{remark}
\newtheorem*{rmk}{Remark}
\newtheorem*{rmks}{Remarks}


% ---------- Symbol Macros ----------
\newcommand{\bra}[1]{\mathopen{}\left(#1\right)}
\newcommand{\sbra}[1]{\mathopen{}\left[#1\right]}
\newcommand{\cbra}[1]{\mathopen{}\left\{#1\right\}}
\newcommand{\abra}[1]{\mathopen{}\left\langle#1\right\rangle}
\newcommand{\norm}[1]{\mathopen{}\left\lVert#1\right\rVert}
\newcommand{\inp}[2]{\mathopen{}\left\langle#1,#2\right\rangle}
\newcommand{\abs}[1]{\mathopen{}\left|#1\right|}
\newcommand{\rest}[2]{\mathopen{}\left.#1\right|_{#2}}

% \newcommand{\R}{\mathbb{R}}
% \newcommand{\N}{\mathbb{N}}
% \newcommand{\Z}{\mathbb{Z}}
% \newcommand{\C}{\mathbb{C}}
% \newcommand{\Q}{\mathbb{Q}}
\newcommand{\R}{\mathbf{R}}
\newcommand{\N}{\mathbf{N}}
\newcommand{\Z}{\mathbf{Z}}
\newcommand{\C}{\mathbf{C}}
\newcommand{\Q}{\mathbf{Q}}

\renewcommand{\d}{\mathrm{d}}
\newcommand{\dx}{\mathrm{d}x}
\newcommand{\eps}{\varepsilon}
% \renewcommand{\epsilon}{\varepsilon}
% \renewcommand{\phi}{\varphi}
\newcommand{\E}{\mathbb{E}}
\renewcommand{\P}{\mathcal{P}}
\renewcommand{\L}{\mathcal{L}}

\newcommand{\T}{\mathrm{T}}
\DeclareMathOperator{\spn}{span}
\DeclareMathOperator{\im}{im}
\DeclareMathOperator{\tr}{tr}
\DeclareMathOperator{\rank}{rank}


\begin{document}
\thispagestyle{empty}


\begin{lem}
    For any $A\subseteq\R$ and any bounded open interval $I$,
    \[\abs A=\abs{A\cap I}+\abs{A\cap I^c}.\]
\end{lem}

\begin{proof}
    $A=(A\cap I)\cup(A\cap I^c)\implies\text{LHS}\le\text{RHS}$. Assume that $\text{LHS}<\text{RHS}$, i.e., $\exists\{I_k\}$ s.t.
    \[A\subseteq\bigcup_{k=1}^\infty I_k,\quad \sum_{k=1}^\infty\ell(I_k)<\abs{A\cap I}+\abs{A\cap I^c}.\]

    Split each $I_k$ into one open interval inside $I$, collected in a new $\{I_k\}$, and two open intervals outside $I$, collected in $\{J_k\}$. Then
    \begin{equation}\label{eq1}
        A\backslash\{a,b\}\subseteq\bra{\bigcup_{k=1}^\infty I_k}\cup\bra{\bigcup_{k=1}^\infty J_k},\quad I_k\in I,J_k\in I^c.
    \end{equation}
    \begin{equation}\label{eq2}
        \sum_{k=1}^\infty\ell(I_k)+\sum_{k=1}^\infty\ell(J_k)<\abs{A\cap I}+\abs{A\cap I^c}.
    \end{equation}
    Add $(a-\eps,a+\eps)$ and $(b-\eps,b+\eps)$ to $\{J_k\}$. (\ref{eq1}) becomes
    \begin{equation}\label{eq3}
        A\subseteq\bra{\bigcup_{k=1}^\infty I_k}\cup\bra{\bigcup_{k=1}^\infty J_k},
    \end{equation}
    and $\eps$ is small enough s.t. (\ref{eq2}) still holds.
    
    Analyze (\ref{eq1}) and (\ref{eq3}). We have
    \[A\cap I\subseteq\bigcup_{k=1}^\infty I_k,\quad A\cap I^c\subseteq\bigcup_{k=1}^\infty J_k,\]
    contradicting (\ref{eq2}).
\end{proof}

\begin{prob}
Prove that for all $A\subseteq\R$,
\[\abs A=\lim_{t\to\infty}\abs{A\cap(-t,t)}.\]
\end{prob}

\begin{proof}
    The case where $\abs A=0$ is trivial.

    Now suppose $\abs A\in\R^+$. Let $\{I_k\}$ be s.t.
    \begin{equation}\label{eq4}
        A\subseteq\bigcup_{k=1}^\infty I_k,
    \end{equation}
    where each $I_k$ is bounded and
    \[\abs A\le\sum_{k=1}^\infty\ell(I_k)<\abs A+\eps.\]
    Then $\exists n$ s.t.
    \[\sum_{k=1}^n\ell(I_k)>\abs A-\eps.\]
    $\exists t$ s.t. $\bigcup_{k=1}^nI_k\subseteq(-t,t)$. Now by (\ref{eq4}), we have
    \[A\cap(-t,t)^c\subseteq\bigcup_{k=n+1}^\infty I_k.\]
    \[\abs{A\cap(-t,t)^c}\le\sum_{k=n+1}^\infty\ell(I_k)<(\abs A+\eps)-(\abs A-\eps)=2\eps.\]

    Now suppose $\abs A=\infty$. WLOG, suppose $A\cap\Z=\emptyset$; otherwise consider $A\backslash\Z$ instead of $A$. By repeatedly using the lemma, we can prove that
    \[\abs A=\sum_{n\in\Z}\abs{A\cap(n,n+1)}.\]
    Because $\abs A=\infty$, $\exists m$ s.t.
    \[\sum_{n=-m}^m\abs{A\cap(n,n+1)}>C.\]
    Hence
    \[\abs{A\cap(-m,m)}=\sum_{n=-m}^{m-1}\abs{A\cap(n,n+1)}>C.\]
    Here the first equality follows from repeatedly using the lemma.
\end{proof}



\end{document}